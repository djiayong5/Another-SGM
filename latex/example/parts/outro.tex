\startconclusionpage

В работе предложен метод, основанный на обучении с подкреплением, позволяющий решать скалярную задачу оптимизации с наличием вспомогательных критериев, свойства которых заранее не известны. Метод повышает производительность эволюционного алгоритма, выбирая вспомогательную функцию приспособленности, наиболее выгодную на данном этапе оптимизации. Предлагаемый метод автоматической настройки эволюционных алгоритмов, основанный на обучении с подкреплением, а также сама постановка модифицированной задачи скалярной оптимизации, возникшей в ходе генерации тестов к олимпиадным задачам по программированию, отличаются новизной.

Проведен эксперимент по решению модельной задачи, в которой на различных этапах оптимизации выгодны различные вспомогательные функции приспособленности. Предлагаемый метод успешно справился с задачей динамического выбора наиболее эффективных ФП. Применение метода позволило стабильно выращивать особи с максимально возможным значением ФП за фиксированное число поколений использованного генетического алгоритма. Также в ходе другого эксперимента подтверждено, что предлагаемый метод игнорирует мешающие ФП, оптимизация по которым может приводить к замедлению роста целевой ФП.

Проведено сравнение предлагаемого метода с методами многокритериальной оптимизации. На примере модельной задачи H-IFF экспериментально показано, что разработанный метод позволяет получить такой же результат, какой может быть получен с использованием наиболее эффективного метода многокритериальной оптимизации. В ходе этого эксперимента также установлено, что применение предлагаемого метода к настройке эволюционной стратегии позволяет выращивать за фиксированное число поколений особи с максимальной приспособленностью в 100\% запусков большинства рассмотренных вариантов эволюционных стратегий, в то время как с помощью обычной эволюционной стратегии без настройки не удается вырастить идеальную особь ни в одном из запусков. Показано, что алгоритмы многокритериальной оптимизации, в отличие от предлагаемого метода, не всегда применимы для решения поставленной задачи, так как их целью является оптимизация всех критериев, что затрудняет их использование в случае, когда некоторые вспомогательные функции приспособленности не коррелируют с целевой. В этом случае разрабатываемый метод позволяет получать лучшие возможные особи в 92\% запусков, в то время как с помощью алгоритма многокритериальной оптимизации PESA-II ни в одном запуске не удалось вырастить особь с максимальной приспособленностью.

Таким образом, экспериментально подтверждено, что разработанный метод соответствует предъявляемым ему требованиям. При наличии помимо целевой ФП хотя бы одной вспомогательной ФП (в том числе мешающей) эффективность работы эволюционного алгоритма под управлением данного метода не ниже, чем эффективность работы того же эволюционного алгоритма при использовании только целевой ФП. При наличии помимо целевой ФП хотя бы одной вспомогательной ФП, оптимизация по которой приводит к более быстрому росту целевой ФП, чем оптимизация по собственно целевой функции, эффективность работы эволюционного алгоритма под управлением данного метода превышает эффективность работы того же эволюционного алгоритма при работе только с целевой ФП. Метод обладает способностью переключаться с одной функции приспособленности на другую, если в процессе оптимизации первая из них перестает быть эффективной.

По тематике данной работы была опубликована статья в Научно-техническом вестнике информационных технологий, механики и оптики \cite{vestnik}. Материал, посвященный предлагаемому методу, опубликован в тезисах к конференции International Conference on Machine Learning and Applicaltions "ICMLA-2011" \cite{strings}. Доклад, сделанный по результатам данной работы, был признан лучшим научно-исследовательским докладом студента на I Всероссийском Конгрессе Молодых Ученых. Также результаты данной работы были отражены в докладе на Всероссийской научной конференции по проблемам информатики "СПИСОК-2012".
