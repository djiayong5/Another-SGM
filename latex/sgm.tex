\documentclass[14pt]{article} % use larger type; default would be 10pt
\usepackage[utf8]{inputenc} % set input encoding (not needed with XeLaTeX)
\usepackage[english,russian]{babel}
\usepackage{geometry} % to change the page dimensions
\geometry{a4paper} % or letterpaper (US) or a5paper or....
\usepackage{graphicx} % support the \includegraphics command and options
\usepackage{amssymb,amsfonts,amsmath,mathtext,cite,enumerate,float} 
\usepackage{algorithm}
\usepackage{algpseudocode}


\algrenewcommand\algorithmicrequire{\textbf{Precondition:}}
\algrenewcommand\algorithmicensure{\textbf{Postcondition:}}
\AtBeginDocument{\renewcommand{\thesection}{}}
\AtBeginDocument{\renewcommand{\thesubsection}{}}

\title{Semi-global matching}
\author{Temp}
\date{}

\begin{document}
\maketitle

Алгоритм построения карты:
\begin{itemize}
 \item Подсчет matching costs $C(p,d), d \in [0, d_max]$. На данном шаге возможны предварительные фильтры входного изображения (mean, rank,...)
 \item Поиск оптимальной карты стереоглубины на основе стоимостей. 
 \item Постпроцессинг
\end{itemize} 


Matching costs and filters: AD,SAD,BT, NCC, MNCC, ZSAD, BilSub, Log, Rank, SoftRank, Census, Ordina, MIl \\
Census: $ p \to (f_{(-n)(-m)}, f_{(-n+1)(-m)}, ...,,  f_{nm})| p_{ij} \in N_p,  f_{ij} = T \| p_{ij} > p \| $

Stereo Algorithms:
\begin{itemize}
\item Energy \\
    Обычно решается задача минимизации функции энергии. \\
     Пример: $E = E_data  + lamda * E_{smooth} $,  где  \\
     $ E_{data} = \sum_p C(p, D_p) $ -  сумма попиксельных стоимостей сопоставления \\
      $E_{smooth} = \sum_{p} \sum_{q \in N_p}(P_1 T \|D_p-D_q\| + P_2 T\|D_p -D_q\|)$ -  штрафы за изменение значения диспаратности\\
   Есть два похода к решению данной задачи: глобальная оптимизация на всем изображении (2D, долго) и оптимизация построчно с помощью динамики (1D). Hirschmuller  предложил оптимизировать динамикой сразу в нескольких направлениях. Затем складывать стоимости и выбирать вариант с минимальной стоимостью.
\item Local window for optimizing
\end{itemize}

SGM
\begin{itemize}
 \item Matching cost: MI 
 \item Stereo algorithm: динамика в 8 направлениях
\end{itemize} 

Существующие улучшения:
\begin{itemize}
 \item Matching cost: hmi - hierarhical mi. Сначала считаем mi для картинки 1/16, потом увеличиваем карту диспаратности в 2 раза и считаем для картинки 1/8 только одну итерацию... Тогда наши затраты $1 +1/2^3 + 1/4^3 + 1/8^3 + 3 / 16^3  \approx 1.14$ вместо 3.  Выйгрыш только в производительности 
 \item Matching cost: Census, bilSub. \\ Дало лучшие результаты, относительно других matching costs: большая устойчивость к изменениям одной из фотографий.
 \item Tiling. Разбиение больших фотографий на множество маленьких картинок, подсчет карт стереоглубины для них, а затем склеивание. При склеивании считать новую диспаратность как взвешенную сумму.
 \item Postprocessing: Consistency check \\
 Проверка правильности построения, с помощью повторного запуска алгоритм с поменянными местами изображениями
\item Postprocessing: Intensity consistent disparity selection \\
 Изначально плохое качество работы на картинках с untextured background.  TODO не входит в рамки изучаемой темы
\item Postprocessing: Интерполяция неизвестных пикселей TODO не входит в рамки изучаемой темы
\item Stereo Algorithm: use $P_2 = \frac{P'_2}{|I_{bp} - I_{bq}|}$
]item Stereo Algorithm: eSgm \\
 Улучшение потребления памяти SGM засчет другого обхода динамики. Результат обработки эквивалентен стандартному алгоритму. $ O( w \times w \times d_{max})$ vs $O(w \times h + w \times d_{max}) $ Также позволило составлять матрицу доверия карте диспаратности (confidence), появились потери в скорости. Рационально использовать для $ d_max > 100 $ 
\end{itemize} 
\end{document}
