\startprefacepage

Задача построения карты глубины по нескольким изображениям является основополагающей в области стереозрения. Восстановление 3D$-$модели во всех вариациях задач сводится к восстановлению карты глубины. Эта фундаментальная проблема была и остается предметом множественных исследований и споров. На данный момент существует большое количество подходов к решению, но все делятся на две категории: локальные и глобальные. Локальные алгоритмы отличаются меньшей точностью и большей производительностью, в то время как глобальные дают очень хорошие результаты, но неприемлемы для приложений реального времени. Алгоритм Semi-Global matching, разработанный Хейко Хиршмюллером, позволяет получить приемлемый результат по качеству и времени. Тем не менее, алгоритм SGM не позволяет обрабатывать изображения высокого разрешения в реальном времени.

В данной работе исследуется возможность ускорения Semi$-$Global matching с помощью использования приблизительной карты глубины. Данный подход позволит уменьшить время работы алгоритма Semi-Global Matching за счёт уменьшения возможных значений диспаратности в каждой точке. В дальнейшем мы сможем использовать комбинацию быстрого локального алгоритма и разработанного метода, такая комбинация сравнима по качеству со стандартным алгоритмом SGM, но превосходит по производительности.